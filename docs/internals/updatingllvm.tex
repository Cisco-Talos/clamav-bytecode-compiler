\chapter{Updating LLVM}
\section{Update LLVM from upstream SVN}
% TODO: use git when LLVM will have official git mirror
\begin{itemize}
\item \verb+cd+ into the git-svn dir of upstream LLVM
\item Update LLVM \footnote{this may require updating the svn-authors file}:
\begin{verbatim}
$ cd llvm
$ git svn fetch
$ git svn rebase --local
\end{verbatim}
\item Update clang:
\begin{verbatim}
$ cd clang
$ git svn fetch
$ git svn rebase --local
\end{verbatim}
\item Build it:
\begin{verbatim}
$ cd ../obj && ../llvm/configure --enable-optimized
$ make -j8
\end{verbatim}
\item All tests must pass before merging to clamav: \verb+make check-all+
\item (Optional) Build ClamAV with clang/x86 backend to test that the C frontend
works:
\begin{verbatim}
$ cd /path/to/clamavsrc
$ ./configure CC=/path/to/clambc-compiler/obj/Release/bin/clang
$ make -j4
$ make check -j4
\end{verbatim}
\end{itemize}

\section{Merging LLVM to ClamAV bytecode compiler}
Use the \verb+merge-new.sh+ script in the bytecode compiler repository.
If there are no conflicts then the script takes care of merging, and comitting
and.

If there are conflicts, the script will stop, and output an error message about
the failed merge.

Fix the conflicts by using \verb+git mergetool+, then
commit the result using \verb+git commit+.

Note that if llvm merge failed, clang is not merged either, so you should resume
the merge of clang (easiest is to just rerun the script).

Then run \verb+make check-all+ for the compiler too.

Note: the script is now doing normal merges (i.e. unsquashed), to visualize just
"our" history use git log --first-parent

\section{Merging LLVM to ClamAV (libclamav)}
Update llvm remote: \verb+git remote update llvm-upstream+.

Use the script \verb|libclamav/c++/merge.sh| as above, from root of ClamAV
source directory, there will be delete/modify conflicts.

Next run the script \verb|libclamav/c++/strip-llvm.sh|, from the
\verb|libclamav/c++| directory, and see if there are any
unneeded dirs left in LLVM. If there are, update the strip script, and rerun it.
Now resolve any merge conflicts, commit the merge, and tag it as instructed by
merge.sh.

Regenerate configure with autoconf 2.65:
\begin{itemize}
\item \verb+cd llvm/autoconf+
\item \verb+sed -i '/Your/d' AutoRegen.sh+
\item \verb+./AutoRegen.sh+
\item \verb+git checkout AutoRegen.sh+
\item \verb+cd ..; git add configure; git add include/llvm/Config/config.h.in+
\end{itemize}

After the merge is complete, update the build files (if needed):
\begin{itemize}
\item do a Debug build of upstream LLVM
\item Run \verb|libclamav/c++/GenList.pl /path/to/llvm-objdir >out|
\item Copy the \_SOURCES definitions from \verb+out+ to
\verb|libclamav/c++/Makefile.am|
\item Run automake in \verb|libclamav/c++|
\item Update the autogenerated files
\item Build ClamAV
\item Update to latest LLVM API (if needed)
\item Build ClamAV
\item Update win32 proj files: \verb+win32/update-win32.pl --regen+
\end{itemize}

To update the autogenerated files:
\begin{itemize}
\item Configure ClamAV in maintainer mode
\footnote{Note that this must be a srcdir == objdir build}:\\
\verb|./configure --enable-maintainer-mode|
\item Build it:\\
\verb|make -j8|
\item If tblgen fails to build, review the list of files in
\verb+tblgen_SOURCES+
\item Review what files changed files (probably .inc and .gen files):\\
\verb|git status|
\item Commit the result:\\
\verb|git commit -a -m "Update autogenerated files after LLVM import"|
\item Fully clean the build dir
\footnote{Be careful to run this inside the ClamAV source dir, and not some other git repository}:\\
\verb|git clean -xfd|
\item Test a normal (non-maintainer build, can be objdir != srcdir):\\
\verb|./configure && make && make check|
\end{itemize}

Run \verb+make check+ from top-level builddir, this will run the LLVM tests too,
make sure all of them pass.

Build ClamAV with \verb+--enable-all-jit-targets+ to test that all supported JIT
targets build.
