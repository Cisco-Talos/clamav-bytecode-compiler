\chapter{Bytecode libclamav hooks}
\section{Logical Signature hooks}
\section{PE hooks}

\section{Adding a new hook}
A bytecode hook consists of the following:
\begin{itemize}
\item special global variables mapped to clamav internal structures,
\item bytecode invoked at certain points in libclamav
\item bytecode APIcalls specific to the hook
\end{itemize}

\subsection{Adding new special globals for hooks}
In the bytecode there are several special global variables named
\verb+__clambc_*+, which are mapped to libclamav internal variables.

These are globals from the bytecode's point of view to make bytecode writing
easier, but they are not real globals in libclamav (it wouldn't be threadsafe).
Instead in libclamav these "special globals" are stored in \verb+struct cli_bc_ctx.hooks+,
and the JIT/interpreter inserts special code to access fields of this struct as
if they were globals.

Steps to add a new global to the bytecode compiler:
\begin{itemize}
\item Choose a unique name for the global
(have a look at \verb+clang/lib/Headers/bytecode_api.h+)

\item Add a new value to \verb+enum bc_global+ in \verb+ClamBC/clambc.h+ named
\verb+GLOBAL_+ followed by the uppercase name of the global. Make sure you add a
new global before \verb+_LAST_GLOBAL+, and don't change the order of the other
enum values (this ensures that bytecodes that don't use the new global continue
to work properly on old versions of libclamav that don't have the new global).

\item Declare the global's name in \verb+ClamBC/ClamBCModule.cpp+: 
 
\verb+globalsMap["__clambc_<name>"] = GLOBAL_<NAME>;+ where \verb+<name>+
and \verb+<NAME>+ are the lowercase/uppercase names of the global.

\item Declare the new global in \verb+clang/lib/Headers/bytecode_api.h+, order
of declaration  of globals doesn't matter here. The global must be declared as
\verb+extern const+ and named \verb+__clambc_+ followed by the lowercase name of
the global.

\item Run \verb+./sync_clamav.sh+ to generate \verb+bytecode_api_decl.c.h+,
\verb+bytecode_api_impl.h+, \verb+bytecode_hooks.h+.
\end{itemize}

Steps to add a new global to libclamav (needed if you add to compiler):
\begin{itemize}
\item In \verb+libclamav/bytecode.c:cli_bytecode_context_alloc()+ initialize the
field of \verb+ctx->hooks+ corressponding to the new global
\item Set the field coresponding to the global in the struct \verb+ctx->hooks+
in one of the API hooks, or introduce a new API hook that sets it.
\item Note that the pointer set must be valid during the entire execution of the
bytecode.
\end{itemize}

\subsection{Adding new bytecode APIs}
Bytecode APIs are external function calls from the bytecode into special
entrypoints in libclamav.

To add a new API follow these steps:
\begin{itemize}
\item Add the prototype for the new API to
\verb+clang/lib/Headers/bytecode_api.h+, inside \verb+#ifdef __CLAMBC__+
\item Run \verb+./sync_clamav.sh+ to synchronize with libclamav
\item Implement the new \verb+cli_bcapi_+ in \verb+libclamav/bytecode_api.c+
\item You can store values in fields of \verb+ctx+, which is a hidden parameter,
    not accessible from bytecode.
\item You can introduce new fields in \verb+ctx+ if needed to implement the API
\item Do validation on input parameters, and any necessary security checks in
the implementation of the API
\item Create a new test in \verb+examples/in/+, with the extension \verb+.o1.c+,
    and update \verb+sync_clamav.sh+ to copy it to \verb+unit_tests/input+
\item Add a new testcase to \verb+unit_tests/check_bytecode.c+:
  \begin{itemize}
  \item Add a new \verb+test_+ function, and add it to the testcase with
  \verb+tcase_add_test+
  \item Call \verb+cl_init+ and \verb+runtest+ similar to other existing unit
  tests, but change the filename to the newly added unittest's name
  \item Run make check, make sure it passes
  \end{itemize}
\end{itemize}

