\chapter{Tutorial}
\section{Short introduction to the bytecode language}
\subsection{Types, variables and constants}
\subsection{Arrays and pointers}
\subsection{Arithmetics}
\subsection{Functions}
\subsection{Control flow}
\subsection{Common functions}

\section[Writing logical signatures]{Writing logical signature bytecodes}
\footnote{See \prettyref{sec:lsigs} for more details about logical signatures in
bytecode.}
Logical signatures can be used as triggers for executing bytecode.
However, instead of describing a logical signature as a \verb+.ldb+ pattern, you use (simple) C code which is later
translated to a \verb+.ldb+-style logical signature by the ClamAV Bytecode Compiler.

A bytecode triggered by a logical signature is much more powerful than a logical signature itself:
you can write complex algorithmic detections, and use the logical signature as a \emph{filter} (to speed up matching).
Thus another name for ``logical signature bytecodes'' is ``algorithmic detection bytecodes''.
The detection you write in bytecode has read-only access to the file being scanned and its metadata
(PE sections, EP, etc.).

\subsection{Structure of a bytecode for algorithmic detection}
Algorithmic detection bytecodes are triggered when a logical signature matches.
They can execute an algorithm that determines whether the file is infected and
with which virus.

A bytecode can be either algorithmic or an unpacker (or other hook), but not both.

It consists of:
\begin{itemize}
 \item Definition of virusnames used in the bytecode
 \item Pattern definitions (for logical subexpressions)
 \item The logical signature as C function: \verb+bool logical_trigger(void)+
 \item The \verb+int entrypoint(void)+ function which gets executed when the logical signature matches
 \item (Optional) Other functions and global constants used in \verb+entrypoint+
\end{itemize}

The syntax for defining logical signatures, and an example is described in \prettyref{sec:lsigstut}.

The function \verb+entrypoint+ must report the detected virus by calling \verb+foundVirus+ and returning \verb+0+.
It is recommended that you always return \verb+0+, otherwise a warning is shown and the file is considered clean.
If \verb+foundVirus+ is not called, then ClamAV also assumes the file is clean.

\subsection{Virusnames}
Each logical signature bytecode must have a virusname prefix, and one or more virusnames.
The virusname prefix is used by the SI to ensure unique virusnames (a unique number is appended for duplicate prefixes).

\begin{program}
\begin{lstlisting}
/* Prefix, used for duplicate detection and fixing */
VIRUSNAME_PREFIX("Trojan.Foo")
/* You are only allowed to set these virusnames as found */
VIRUSNAMES("A", "B")
/* File type */
TARGET(2)
\end{lstlisting}
\caption{Declaring virusnames}
\label{prg:virname}
\end{program}

In \prettyref{prg:virname} 3 predefied macros are used:
\begin{itemize}
 \item \verb+VIRUSNAME_PREFIX+ which must have exactly one string argument
 \item \verb+VIRUSNAMES+ which must have one or more string arguments
 \item \verb+TARGET+ which must have exactly one integer argument
\end{itemize}

In this example, the bytecode could generate one of these virusnames:
\verb+Trojan.Foo.A+, or \verb+Trojan.Foo.B+, by calling
\verb+foundVirus("A")+ or \verb+foundVirus("B")+ respectively (notice that the prefix is not part of these calls).

\subsection{Patterns}
Logical signatures use \verb+.ndb+ style patterns, an example on how to define these
is shown in \prettyref{prg:patterns}.
\begin{program}
\begin{lstlisting}
SIGNATURES_DECL_BEGIN
DECLARE_SIGNATURE(magic)
DECLARE_SIGNATURE(check)
DECLARE_SIGNATURE(zero)
SIGNATURES_DECL_END

SIGNATURES_DEF_BEGIN
DEFINE_SIGNATURE(magic, "EP+0:aabb")
DEFINE_SIGNATURE(check, "f00d")
DEFINE_SIGNATURE(zero, "ffff")
SIGNATURES_END
\end{lstlisting}
\caption{Declaring patterns}
\label{prg:patterns}
\end{program}

Each pattern has a name (like a variable), and a string that is the hex pattern itself.
The declarations are delimited by the macros \verb+SIGNATURES_DECL_BEGIN+, and \verb+SIGNATURES_DECL_END+.
The definitions are delimited by the macros \verb+SIGNATURES_DEF_BEGIN+, and \verb+SIGNATURES_END+.
Declarations must always come before definitions, and you can have only one
declaration and declaration section!
(think of declaration like variable
declarations, and definitions as variable assignments, since that what they are under the hood).
The order in which you declare the signatures is the order in which they appear in the generated
logical signature.

You can use any name for the patterns that is a valid record field name in C,
and doesn't conflict with anything else declared.

After using the above macros, the global variable \verb+Signatures+ will have two new fields:
\verb+magic+, and \verb+zero+.
These can be used as arguments to the functions \verb+count_match()+, and \verb+matches()+
anywhere in the program as shown in \prettyref{prg:counting}:
\begin{itemize}
 \item \verb+matches(Signatures.match)+ will return true when the \verb+match+ signature matches (at least once)
 \item \verb+count_match(Signatures.zero)+ will return the number of times the \verb+zero+ signature matched
 \item \verb+count_match(Signatures.check)+ will return the number of times the \verb+check+ signature matched
\end{itemize}

The condition in the \verb+if+ can be interpreted as: if the \verb+match+ signature has matched at least once,
and the number of times the \verb+zero+ signature matched is higher than the number of times the \verb+check+ signature matched,
then we have found a virus \verb+A+, otherwise the file is clean.

\begin{program}
\begin{lstlisting}
int entrypoint(void)
{
  if (matches(Signatures.match) && count_match(Signatures.zero) > count_match(Signatures.check))
    foundVirus("A");
  return 0;
}
\end{lstlisting}
\caption{Using patterns}
\label{prg:counting}
\end{program}

\subsection{Single subsignature}
\label{sec:lsigstut}
The simplest logical signature is like a \verb+.ndb+ signature: a virus name, signature target, 0 as logical expression
\footnote{meaning that subexpression 0 must match}, and a \verb+ndb+-style pattern.

The code for this is shown in \prettyref{prg:singlesig}
\begin{program}
\lstinputlisting{../../test/examples/in/lsig_simple.o1.c}
\caption{Single subsignature example}
\label{prg:singlesig}
\end{program}

The logical signature (created by the compiler) looks like this:
\verb+Trojan.Foo.{A};Target:2;0;aabb+

Of course you should use a \verb+.ldb+ signature in this case when all the processing in \verb+entrypoint+
is only setting a virusname and returning.
However, you can do more complex checks in \verb+entrypoint+, once the bytecode was triggered by the \verb+logical_trigger+

In the example in \prettyref{prg:singlesig} the pattern was used without an anchor; such a pattern matches at any offset.
You can use offsets though, the same way as in \verb+.ndb+ signatures, see \prettyref{prg:multisig} for an example.

\subsection{Multiple subsignatures}
An example for this is shown in \prettyref{prg:multisig}.
Here you see the following new features used:
\footnote{In case of a duplicate virusname the
prefix is appended a unique number by the SI}
\begin{itemize}
 \item Multiple virusnames returned from a single bytecode (with common prefix)
 \item Multiple subsignatures, each with a name of your choice
 \item A pattern with an anchor (\verb|EP+0:aabb|)
 \item More subsignatures defined than used in the logical expression
\end{itemize}

The logical signature looks like this:

\noindent
{\footnotesize
\verb/Trojan.Foo.{A,B};Target:2;(((0|1|2)=42,2)|(3=10));EP+0:aabb;ffff;aaccee;f00d;dead/
}

Notice how the subsignature that is not used in the logical expression (number 4, \verb+dead+)
is used in \verb+entrypoint+ to decide the virus name.
This works because ClamAV does collect the match counts for all subsignatures (regardless if they are used or not in
a signature). The \verb+count_match(Signatures.check2)+ call is thus a simple memory read of the count already determined by ClamAV.

Also notice that comments can be used freely: they are ignored by the compiler. You can use either C-style multiline comments
(start comment with \verb+/*+, end with \verb+*/+), or C++-style single-line comments
(start comment with \verb+//+, automatically ended by newline).

\begin{program}
\lstinputlisting{../../test/examples/in/lsig.o1.c}
\caption{Multiple subsignatures}
\label{prg:multisig}
\end{program}

\subsection{W32.Polipos.A detector rewritten as bytecode}
\subsection{Virut detector in bytecode}
\section{Writing regular expressions in bytecode}
ClamAV only supports a limited set of regular expressions in \verb+.ndb format+ : wildcards.
The bytecode compiler allows you to compile fully generic regular expressions to bytecode directly.
When libclamav loads the bytecode, it will compile to native code (if using the JIT), so it should offer quite good performance.

The compiler currently uses \verb+re2c+ to compile regular expressions to C code, and then compile that to bytecode.
The internal workings are all transparent to the user: the compiler automatically uses \verb+re2c+ when needed, and \verb+re2c+
is embedded in the compiler, so you don't need to install it.

The syntax of regular expressions are similar to the one used by \verb+POSIX regular expressions+, except you have to quote
literals, since unquoted they are interpreted as regular expression names.

\subsection{A very simple regular expression}

Lets start with a simple example, to match this POSIX regular expression:
\verb+eval([a-zA-Z_][a-zA-Z0-9_]*\.unescape+.

See \prettyref{prg:reg1} \footnote{This omits the virusname, and logical signature declarations}.
\begin{program}
\lstinputlisting{../../test/examples/in/reg1.o1.c}
\caption{Simple regular expression example}
 \label{prg:reg1}
\end{program}

There are several new features introduced here, here is a step by step breakdown:
\begin{description}
 \item \verb+REGEX_SCANNER+ this declares the data structures needed by the regular expression matcher
 \item \verb+seek(0, SEEK_SET)+ this sets the current file offset to position \verb+0+, matching will start at this position.
For offset 0 it is not strictly necessary to do this, but it serves as a reminder that you might want to start matching somewhere, that is not necessarily 0.
 \item \verb+ for(;;) { REGEX_LOOP_BEGIN+ this creates the regular expression matcher main loop. It takes the current file byte-by-byte \footnote{it is not really
reading byte-by-byte, it is using a buffer to speed things up} and tries to match one of the regular expressions.
 \item \verb+/*!re2c+ This mark the beginning of the regular expression description. The entire regular expression block is a C comment, starting with \verb+!re2c+
 \item \verb+ANY = [^];+ This declares a regular expression named \verb+ANY+ that matches any byte.
 \item \verb+"eval("[a-zA-Z_][a-zA-Z_0-9]*".unescape" {+ This is the actual regular expression.
\begin{description}
 \item \verb+"eval("+ This matches the literal string \verb+eval(+.
Literals have to be placed in double quotes \verb+"+ here, unlike in POSIX regular expressions or PCRE.
If you want case-insensitive matching, you can use \verb+'+.
 \item \verb+[a-zA-Z_]+ This is a character class, it matches any lowercase, uppercase or \verb+_+ characters.
 \item \verb+[a-zA-Z_0-9]*"+ Same as before, but with repetition. \verb+*+ means match zero or more times, \verb|+| means match one or more times, just like in POSIX
regular expressions.
 \item \verb+".unescape"+ A literal string again
 \item \verb+{+ start of the \emph{action} block for this regular expression. Whenever the regular expression matches, the attached C code is executed.
\end{description}
 \item \verb+long pos = REGEX_POS;+ this determines the absolute file offset where the regular expression has matched. Note that because
the regular expression matcher uses a buffer, using just \verb+seek(0, SEEK_CUR)+ would give the current position of the end of that buffer, and
not the current position during regular expression matching. You have to use the \verb+REGEX_POS+ macro to get the correct position.
 \item \verb+debug(...)+ Shows a debug message about what was found and where. This is extremely helpful when you start writing regular expressions, and
nothing works: you can determine whether your regular expression matched at all, and if it matched where you thought it would.
There is also a \verb+DEBUG_PRINT_MATCH+ that prints the entire matched string to the debug output.
Of course before publishing the bytecode you might want to turn off these debug messages.
 \item \verb+}+ closes the \emph{action} block for this regular expression
 \item \verb+ANY { continue; }+ If none of the regular expressions matched so far, just keep running the matcher, at the next byte
 \item \verb+*/+ closes the regular expression description block
 \item \verb+}+ closes the \verb+for()+ loop
\end{description}

You may have multiple regular expressions, or declare multiple regular expressions with a name, and use those
names to build more complex regular expressions.

\subsection{Named regular expressions}


\section{Writing unpackers}
\label{sec:unpacker}
\subsection{Structure of a bytecode for unpacking (and other hooks)}
When writing an unpacker, the bytecode should consist of:
\begin{itemize}
 \item Define which hook you use (for example \verb+PE_UNPACKER_DECLARE+ for a PE hook)
 \item An \verb+int entrypoint(void)+ function that reads the current file and unpacks it to a new file
 \item Return 0 from \verb+entrypoint+ if you want the unpacked file to be scanned %TODO: how do we avoid infinite recursion
% here, if the bytecode is invoked again, and unpacks something again? Or mutual recursion...
%TODO: I don't call magicscandesc yet, until a solution is found here
 \item (Optional) Other functions and global constants used by \verb+entrypoint+
\end{itemize}

\subsection{Detecting clam.exe via bytecode}
Example provided by aCaB:
\subsection{Detecting clam.exe via bytecode (disasm)}
Example provided by aCaB:

\label{prg:matchclamexedisasm}
\subsection{A simple unpacker}
\subsection{Matching PDF javascript}

\subsection{YC unpacker rewritten as bytecode}
