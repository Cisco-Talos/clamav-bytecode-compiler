\chapter{ClamAV bytecode language}
The bytecode that ClamAV loads is a simplified form of the LLVM Intermediate Representation, and as such it is language-independent.

However currently the only supported language from which such bytecode can be generated is a simplified form of C
\footnote{In the future more languages could be supported, see the Internals Manual on language frontends}

The language supported by the ClamAV bytecode compiler is a restricted set of C99 with some GNU extensions.

\section{Differences from C99 and GNU C}
\label{sec:diffc}
These restrictions are enforced at compile time:
\begin{itemize}
 \item No standard include files. \footnote{For portability reasons: preprocessed C code is not portable}
 \item The ClamAV API header files are preincluded.
 \item No external function calls, except to the ClamAV API \footnote{For safety reasons we can't allow the bytecode to call arbitrary system functions}
 \item No inline assembly \footnote{This is both for safety and portability reasons}
 \item Globals can only be readonly constants \footnote{For thread safety reasons}
 \item \verb+inline+ is C99 inline (equivalent to GNU C89 extern inline), thus it cannot be used outside of the definition of the ClamAV API,
you should use \verb+static inline+
 \item \verb+sizeof(int) == 4+ always
 \item \verb+sizeof(long) == sizeof(long long) == 8+ always
 \item \verb+ptr_diff_t = int+, \verb+intptr_t = int+, \verb+intmax_t = long+, \verb+uintmax_t = unsigned long+ 
\footnote{Note that a pointer's sizeof is runtime-platform dependent, although at compile time sizeof(void*) == 4, at runtime it can be something else.
Thus you should avoid using sizeof(pointer)}
 \item No pointer to integer casts and integer to pointer casts (pointer arithmetic is allowed though)
 \item No \verb+__thread+ support
 \item Size of memory region associated with each pointer must be known in each function, thus if you pass a pointer to a function,
you must also pass its allocated size as a parameter.
 \item Endianness must be handled via the \verb+__is_bigendian()+ API function call, or via the \verb+cli_{read,write}int{16,32}+ wrappers,
and not by casting pointers
 \item Predefines \verb+__CLAMBC__+
 \item All integer types have fixed width
 \item \verb+main+ or \verb+entrypoint+ must have the following prototype: \verb+int main(void)+, the prototype
\verb+int main(int argc, char *argv[])+ is not accepted %TODO: really reject this, for now its compiled but clamav won't find main when loading
%TODO: The ClamBC backend should default to bigendian, so that such endianness bugs are caught soon, i.e. they will be immediately buggy when executing on x86,
% unless the endian macros are used
\end{itemize}

They are meant to ensure the following:
\begin{itemize}
 \item Thread safe execution of multiple different bytecodes, and multiple instances of the same bytecode
 \item Portability to multiple CPU architectures and OSes: the bytecode must execute on both the libclamav/LLVM JIT where that is supported (x86, x86\_64, ppc, arm?),
and on the libclamav interpreter where that is not supported.
 \item No external runtime dependency: libclamav should have everything needed to run the bytecode, thus no external calls are allowed, not even to libc!
 \item Same behaviour on all platforms: fixed size integers. 
\end{itemize}

These restrictions are checked at runtime (checks are inserted at compile time):
\begin{itemize}
 \item Accessing an out-of-bounds pointer will result in a call to \verb+abort()+
 \item Calling \verb+abort()+ interrupts the execution of the bytecode in a thread safe manner, and doesn't halt ClamAV
\footnote{in fact it calls a ClamAV API function, and not the libc abort function.}.
\end{itemize}

The ClamAV API header has further restriction, see the Internals manual.

Although the bytecode undergoes a series of automated tests (see Publishing chapter in Internals manual), the above restrictions don't guarantee
that the resulting bytecode will execute correctly!
You must still test the code yourself, these restrictions only avoid the most common errors.
Although the compiler and verifier aims to accept only code that won't crash ClamAV, no code is 100\% perfect, and a bug in the verifier
could allow unsafe code be executed by ClamAV.

\section{Limitations}
The bytecode format has the following limitations:
\begin{itemize}
 \item At most 64k bytecode kinds (hooks)
 \item At most 64k types (including pointers, and all nested types)
 \item At most 16 parameters to functions, no vararg functions
 \item At most 64-bit integers
 \item No vector types or vector operations
 \item No opaque types
 \item No floating point
 \item Global variable initializer must be compile-time computable
 \item At most 32k global variables (and at most 32k API globals)
 \item Pointer indexing at most 15 levels deep (can be worked around if needed by using temporaries)
 \item No struct return or byval parameters
 \item At most 32k instructions in a single function
 \item No Variable Length Arrays
\end{itemize}



\section{Logical signatures}
\label{sec:lsigs}
Logical signatures can be used as triggers for executing a bytecode. 
Instead of describing a logical signatures as a \verb+.ldb+ pattern, you use C code which is then
translated to a \verb+.ldb+-style logical signature.

Logical signatures in ClamAV support the following operations:
\begin{itemize}
 \item Sum the count of logical subsignatures that matched inside a subexpression
 \item Sum the number of different subsignatures that matched inside a subexpression
 \item Compare the above counts using the $>,=,<$ relation operators
 \item Perform logical $\&\&, ||$ operations on above boolean values
 \item Nest subexpressions
 \item Maximum 64 subexpressions
\end{itemize}

Out of the above operations the ClamAV Bytecode Compiler doesn't support computing sums of nested subexpressions,
(it does support nesting though).

The C code that can be converted into a logical signature must obey these restrictions:
\begin{itemize}
 \item a function named \verb+logical_trigger+ with the following prototype:
\verb+bool logical_trigger(void)+
 \item no function calls, except for \verb+count_match+ and \verb+matches+
 \item no global variable access (except as done by the above 2 functions internally)
 \item return true when signature should trigger, false otherwise
 \item use only integer compare instructions, branches, integer \emph{add}, logical \emph{and}, logical \emph{or},
logical \emph{xor}, zero extension, store/load from local variables
 \item the final boolean expression must be convertible to disjunctive normal form without negation
 \item the final logical expression must not have more than 64 subexpressions
 \item it can have early returns (all true returns are unified using $||$)
 \item you can freely use comments, they are ignored
 \item the final boolean expression cannot be a \verb+true+ or \verb+false+ constant
\end{itemize}

The compiler does the following transformations (not necessarily in this order):
\begin{itemize}
 \item convert shortcircuit boolean operations into non-shortcircuit ones (since all operands are boolean expressions or local variables,
it is safe to execute these unconditionally)
 \item propagate constants
 \item simplify control flow graph
 \item (sparse) conditional constant propagation
 \item dead store elimination
 \item dead code elimination
 \item instruction combining (arithmetic simplifications)
 \item jump threading
\end{itemize}

If after this transformation the program meets the requirements outlined above, then it is converted to a logical signature.
The resulting logical signature is simplified using basic properties of boolean operations, such as
associativity, distributivity, De Morgan's law.

The final logical signature is not unique (there might be another logical signature with identical behavior), however the boolean part is in a canonical form:
it is in disjunctive normal form, with operands sorted in ascending order.

For best results the C code should consist of:
\begin{itemize}
 \item local variables declaring the sums you want to use
 \item a series of \verb+if+ branches that \verb+return true+, where the \verb+if+'s condition is a single comparison or a logical \emph{and} of comparisons
 \item a final \verb+return false+
\end{itemize}

You can use $||$ in the \verb+if+ condition too, but be careful that after expanding to disjunctive normal form, the number of subexpressions doesn't exceed 64.

Note that you do not have to use all the subsignatures you declared in \verb+logical_trigger+, you can
do more complicated checks (that wouldn't obey the above restrictions) in the bytecode itself at runtime.
The \verb+logical_trigger+ function is fully compiled into a logical signature, it won't be a runtime executed function (hence the restrictions).

\section{Headers and runtime environment}
When compiling a bytecode program, \verb+bytecode.h+ is automatically included, so you don't need to explicitly include it.
These headers (and the compiler itself) predefine certain macros, see \prettyref{apdx:predefined} for a full list.
In addition the following types are defined:
\begin{lstlisting}
typedef unsigned char uint8_t;
typedef char int8_t;
typedef unsigned short uint16_t;
typedef short int16_t;
typedef unsigned int uint32_t;
typedef int int32_t;
typedef unsigned long uint64_t;
typedef long int64_t;
typedef unsigned int size_t;
typedef int off_t;
typedef struct signature { unsigned id } __Signature;
\end{lstlisting}
As described in \prettyref{sec:diffc} the width of integer types are fixed, the above typedefs show that.

A bytecode's entrypoint is the function \verb+entrypoint+ and it's required by ClamAV to load the bytecode.

Bytecode that is triggered by a logical signature must have a list of virusnames and patterns defined.
Bytecodes triggered via hooks can optionally have them, but for example a PE unpacker doesn't need virus
names as it only processes the data.
