\chapter{Usage}
\section{Invoking the compiler}
Compiling is similar to gcc \footnote{Note that the ClamAV bytecode compiler will refuse to compile code it considers insecure}:
\begin{verbatim}
$ /usr/local/clamav/bin/clambc-compiler foo.c -o foo.cbc -O2
\end{verbatim}

This will compile the file \verb+foo.c+ into a file called \verb+foo.cbc+,
that can be loaded by ClamAV, and packed inside a \verb+.cvd+ file.

The compiler by default has all warnings turned on.

Supported optimization levels: \verb+-O0, -O1, -O2, -O3+. \footnote{Currently -O0 doesn't work}
It is recommended that you always compile with at least \verb+-O1+.

Warning options: \verb+-Werror+ (transforms all warnings into errors).

Preprocessor flags:
\begin{description}
 \item[-I <directory>] Searches in the given directory when it encounters a \verb+#include "headerfile"+ directive in the source code, in addition
to the system defined header search directories.
 \item[-D <MACRONAME>=<VALUE>] Predefine given \verb+<MACRONAME>+ to be equal to \verb+<VALUE>+.
 \item[-U <MACRONAME>] Undefine a predefined macro
\end{description}

The compiler also supports some other commandline options (see \verb+clambc-compiler --help+ for a full list),
however some of them have no effect when using the ClamAV bytecode backend (such as the X86 backend options).
You shouldn't need to use any flags not documented above.

\subsection{Compiling C++ files}
Filenames with a \verb+.cpp+ extension are compiled as C++ files, however \verb|clang++| is not yet
ready for production use, so this is EXPERIMENTAL currently.
For now write bytecodes in C.

\section{Running compiled bytecode}
After compiling a C source file to bytecode, you can load it in ClamAV:
\subsection{ClamBC}
ClamBC is a tool you can use to test whether the bytecode loads, compiles, and can execute its entrypoint successfully.
Usage:
\begin{verbatim}
 clambc <file> [function] [param1 ...]
\end{verbatim}

For example loading a simple bytecode with 2 functions is done like this:
\begin{verbatim}
$ clambc foo.cbc
LibClamAV debug: searching for unrar, user-searchpath: /usr/local/lib
LibClamAV debug: unrar support loaded from libclamunrar_iface.so.6.0.4 libclamunrar_iface_so_6_0
LibClamAV debug: bytecode: Parsed 0 APIcalls, maxapi 0
LibClamAV debug: Parsed 1 BBs, 2 instructions
LibClamAV debug: Parsed 1 BBs, 2 instructions
LibClamAV debug: Parsed 2 functions
Bytecode loaded
Running bytecode function :0
Bytecode run finished
Bytecode returned: 0x8
Exiting
\end{verbatim}

\subsection{clamscan, clamd}
You can tell clamscan to load the bytecode as a database directly:
\begin{verbatim}
$ clamscan -dfoo.cbc
\end{verbatim}
Or you can instruct it to load all databases from a directory, then clamscan will
load all supported formats, including files with bytecode, which have the \verb+.cbc+ extension.
\begin{verbatim}
$ clamscan -ddirectory
\end{verbatim}

% TODO: we should describe or point to the security levels here when they get implemented
You can also put the bytecode files into the default database directory of ClamAV
(usually \verb+/usr/local/share/clamav+) to have it loaded automatically from there.
Of course, the bytecode can be stored inside CVD files, too.

\section{Debugging bytecode}
\subsection{``printf'' style debugging}
You can use \verb+debug_print_str+ and \verb+debug_print_uint+ API %TODO: link to API section
calls to print debug messages during the execution of the bytecode.

\subsection{Single-stepping}
%TODO: some glue code missing here: LLVM supports generation of dwarf debug info (including variable and line number info),
%however variable debug info is lost during the most simple transform (mem2reg), and is undergoing a major redesign now,
%and line number debug info is only emited when compiling statically.
%Although line number debug info IS available in the JIT (assuming the loaded bytecode has it, by default it doesn't),
%it currently doesn't emit dwarf debug info at all when JITing.
% So what works is call frame info, i.e. backtraces, and function names in backtraces
If you have GDB 7.0 (or newer) you can single-step \footnote{not yet implemented in libclamav} \footnote{assuming you have JIT support}
during the execution of the bytecode.
\begin{itemize}
 \item Run clambc or clamscan under gdb:
\begin{verbatim}
$ ./libtool --mode=execute gdb clamscan/clamscan
...
(gdb) b cli_vm_execute_jit
Are you sure ....? y
(gdb) run -dfoo.cbc
...
Breakpoint ....

(gdb) step
(gdb) next
\end{verbatim}

You can single-step through the execution of the bytecode, however you can't (yet) print values of individual variables,
you'll need to add debug statements in the bytecode to print interesting values.

\end{itemize}


